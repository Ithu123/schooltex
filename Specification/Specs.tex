\documentclass{scrartcl}

% Packages for document formatting
\usepackage{geometry} % Adjust page margins
\usepackage{parskip} % Add space between paragraphs
\usepackage{hyperref} % Create hyperlinks
\usepackage{listings} % Typeset code listings
\usepackage{marginnote} % Place notes in the margin
\usepackage{enumitem} % Customize lists

% Document metadata
\title{My Documentation Title}
\author{Author Name}
\date{\today}

% Define a command for margin notes
% \newcommand{\cmd}[1]{\marginnote{\texttt{\string#1}}}
\newcommand{\cmd}[1]{\paragraph*{\texttt{\string#1}}}
\newcommand{\cmdtext}[1]{\texttt{\string#1}}
\newcommand{\opt}[1]{\texttt{<\string#1>}}
\begin{document}

\maketitle

\tableofcontents

% \section{Introduction}
% % Your introduction here

% \section{Installation}
% % Instructions for installing your software or package

% \section{Usage}
% % Instructions for using your software or package

% \section{Examples}
% % Code examples or usage scenarios

% \section{FAQ}
% % Frequently asked questions and answers

% \section{Conclusion}
% % Final thoughts or concluding remarks

% Typical commands in the margin
\section*{Commands}
\cmd{\grid[<length>]}: This produces a grid of width textwidth. \texttt{<length>}
is an integer, wich defines the length of the grid in centimeter. If left empty, it will span to the end of the page.

\cmd{\materials[\opt{options}]\{\opt{material1},\opt{material1},...\}}: A Collection of materials assosiated with that
exercise. Following options should be present:
\begin{itemize}
    \item \opt{position}: Choose from: \opt{up}, \opt{down}, \opt{left}, \opt{right}. The materials get placed accordingly.
    \item \opt{material}: Should be possible to use \texttt{\textbackslash image} or any \texttt{table}-command
\end{itemize}
Further Specs:
\begin{itemize}
    \item The Env. should always place the according to the available space, eighther in one line or multiple lines.
    \item The width of each material should be adjustable. 
    \item The material should be seperated by a \texttt{;} , need a command to escape ;.
    \item The Ex. text should flow around the materials.
\end{itemize}


\cmd{\image[\opt{options}]\{\opt{image1}\}}: A non floating single image witch has the following options: 
\begin{itemize}
    \item \opt{width} The width of the image.
    \item \opt{caption} The caption for the image.
    \item \opt{label} A Label for reference.
\end{itemize}




\cmd{\multiimage[\opt{options}]\{\opt{image1},\opt{image2},...\}}: A non floating image command, wich takes one or more 
images, and puts the next to each other. Non-compatible with \texttt{materials}

The image should have the 



% Bibliography if needed
%\bibliographystyle{plain}
%\bibliography{references}
\section*{Enviroments}
\cmd{\Aufgabe[<points>][<symbol>]}:Defines a new exercise.
\begin{itemize}
    \item \opt{points} : Decimal number, declares the points given to each Ex.
    \item \opt{symbol} : Integer, Chooses from diffrent symbols to put a the margin. If empty it defaults to a little sheet with a pen.
\end{itemize}
Please note that for now to change the symbol, the point bracket must be present.


\section*{Thoughts on structure}
\paragraph*{Label and caption:} There should be a function wich takes label and caption
and handels them:
\begin{itemize}
    \item if only a lable is specified, the caption should just read: "figure 1".
\end{itemize}



\end{document}
