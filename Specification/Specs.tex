\documentclass{scrartcl}

% Packages for document formatting
\usepackage{geometry} % Adjust page margins
\usepackage{parskip} % Add space between paragraphs
\usepackage{hyperref} % Create hyperlinks
\usepackage{listings} % Typeset code listings
\usepackage{marginnote} % Place notes in the margin
\usepackage{enumitem} % Customize lists

% Document metadata
\title{My Documentation Title}
\author{Author Name}
\date{\today}

% Define a command for margin notes
% \newcommand{\cmd}[1]{\marginnote{\texttt{\string#1}}}
\newcommand{\cmd}[1]{\paragraph*{\texttt{\string#1}}}
\newcommand{\opt}[1]{\texttt{<\string#1}>}
\begin{document}

\maketitle

\tableofcontents

\section{Introduction}
% Your introduction here

\section{Installation}
% Instructions for installing your software or package

\section{Usage}
% Instructions for using your software or package

\section{Examples}
% Code examples or usage scenarios

\section{FAQ}
% Frequently asked questions and answers

\section{Conclusion}
% Final thoughts or concluding remarks

% Typical commands in the margin
\section*{Commands}
\cmd{\grid[<length>]}: This produces a grid of width textwidth. \texttt{<length>}
is an integer, wich defines the length of the grid in centimeter. If left empty, it will span to the end of the page.
% Bibliography if needed
%\bibliographystyle{plain}
%\bibliography{references}
\section*{Enviroments}
\cmd{Aufgabe[<points>][<symbol>]}Defines a new exercise.
\begin{itemize}
    \item \opt{points}: Decimal number, declares the points given to each Ex.
    \item \opt{symbol}: Integer, Chooses from diffrent symbols to put a the margin. If empty it defaults to a little sheet with a pen.
\end{itemize}
Please note that for now to change the symbol the point bracket must be present

\end{document}
