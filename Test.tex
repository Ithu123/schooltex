\documentclass[a4paper]{scrartcl}
\usepackage{schooltex}
\declarePath{.}
\usepackage{blindtext}
\usepackage[german]{babel}
\setkomafont{captionlabel}{\itshape}
\usepackage{lmodern}
\usepackage[headsepline, footsepline]{scrlayer-scrpage}
\pagestyle{scrheadings}
\usepackage[bottom=2.5cm, top=22mm, right = 2.5cm]{geometry}
\usepackage{hyperref}

\newcommand{\fach}{Physik}

\ifoot{Baier}
\ofoot{\fach}
\cfoot{test}

%\uselengthunit{cm}
\newcommand{\Datum}{xx.xx.23}
\begin{document}
\begin{SVerlauf}{Thema}
10' & Einstieg & Das wäre der Einstieg & & \ze
\end{SVerlauf}


\begin{SVerlauf}{Thema}
    10' & Einstieg & Das wäre der Einstieg & & \ze
\end{SVerlauf}

    \tafel{\kap{Ein Kapitel}
        \uber{Ueberschrift}
    Das ist ein Tafelbild.
    }

\newpage
\header{90}{Schreibzeug}
\begin{Aufgabe}[3]TEs
    \teilaufgabe[1]{testen}
\end{Aufgabe}
\begin{Aufgabe}[3]TEs
    \teilaufgabe[1]{testen}
\end{Aufgabe}
\begin{Aufgabe}[24]
    \blindtext

\end{Aufgabe}
\grid
\blindtext
\multiimage[widtheach = 5cm]{example-image-a,example-image-a}
 
 %\image[source = www.example.de]{Atwood2}
%\image[width = 6cm, caption = Eine Abbildung mit Caption und Quelle, source = www.example.de]{Atwood2}
%\image[width = 6cm]{Atwood2}
%\image[caption = Ohne Quelle{,} aber mit Caption]{Atwood2}
%\multiimage[source = www.expamle.com]{Atwood2,Atwood2, Atwood2, Atwood2, Atwood2}
%\multiimage[caption = Diesmal auch mit Caption , source = www.expamle.com]{Atwood2,Atwood2, Atwood2, Atwood2, Atwood2}
%\multiimage[caption = Mit caption]{Atwood2,Atwood2, Atwood2, Atwood2, Atwood2}
%blindtext
%\image[width = 10cm, source = examplequelle, position = l, label = fig:Label]{example-image-a}
%\blindtext
%In Abbildung \ref{fig:Label} sehen wir, dass wir hier ein Label definiert haben. Ich hoffe es klappt.
%\AB[]{UebungEnergieundLeistung}
%\AB[small]{UebungEnergieundLeistung}
%\AB[smalllandscape]{UebungEnergieundLeistung}
%\newpage


\end{document}