\documentclass[a4paper]{scrartcl}
\usepackage{schooltex}
\declarePath{.}
\usepackage{blindtext}
\usepackage[german]{babel}
\setkomafont{captionlabel}{\itshape}
\usepackage{lmodern}
\usepackage[headsepline, footsepline]{scrlayer-scrpage}
\pagestyle{scrheadings}
\usepackage[bottom=2.5cm, top=22mm, right = 2.5cm]{geometry}
\usepackage{hyperref}

\newcommand{\fach}{Physik}

\ifoot{Baier}
\ofoot{\fach}
\cfoot{}

%\uselengthunit{cm}
\newcommand{\Datum}{xx.xx.23}
\begin{document}


% \begin{SVerlauf}
%     10' & Ein Beispiel & Dies ist ein Beispiel & \lsg & Tafel \\ 
%     10' & Ein zweites Beispiel & Dies ist ein zweites  Beispiel Dies ist ein zweites  Beispiel Dies ist ein zweites  Beispiel Dies ist ein zweites  Beispiel  & \lsg & Tafel \\ 
%     10' & Ein zweites Beispiel & Dies ist ein zweites  Beispiel & \lsg & Tafel  \\ 
%     10' & Ein zweites Beispiel & Dies ist ein zweites  Beispiel & \lsg & Tafel \\ 
% \end{SVerlauf}
\begin{plan}[Titel der Stunde]
10' & Ein Beispiel & dies ist ein Beispiel & \lsg & Tafel \\ 
10' & Ich teste das: &Wenn ich hier jetzt einen riesigen Text hineinschreibe, der im Prinzip die gesamte
Information über die Stunde enthält, wird das dann etwas?  & \lsg & Tafel, Dokumentenkamera \\ 
10' & Ein zweites Beispiel & dies ist ein zweites  Beispiel & \lsg & Tafel  \\ 
10' & Ein zweites Beispiel & dies ist ein zweites  Beispiel & \lsg & Tafel \\ 
\end{plan}


\begin{plan}[Titel der Stunde]
    10' & Ein Beispiel & dies ist ein Beispiel & \lsg & Tafel \\ 
    10' & Ich teste das: &Wenn ich hier jetzt einen riesigen Text hineinschreibe, der im Prinzip die gesamte
    Information über die Stunde enthält, wird das dann etwas?  & \lsg & Tafel, Dokumentenkamera \\ 
    10' & Ein zweites Beispiel & dies ist ein zweites  Beispiel & \lsg & Tafel  \\ 
    10' & Ein zweites Beispiel & dies ist ein zweites  Beispiel & \lsg & Tafel \\ 
\end{plan}

 \begin{Aufgabe}
 teste
 \grid[4]
 \end{Aufgabe}

\begin{Aufgabe}[1] 
    \blindtext
\end{Aufgabe}
\begin{Aufgabe}
    \blindtext
\end{Aufgabe}

\begin{Aufgabe}[5]
    Eine Aufgabe mit Teilaufgabe
    \teilaufgabe[4]{Dies ist eine Teilaufgabe. Sie geht ein bisschen länger, sodass man das Alignment sieht denke ich.}
    \teilaufgabe[3]{Dies ist eine Teilaufgabe. Sie geht ein bisschen länger, sodass man das Alignment sieht denke ich.}
    \teilaufgabe[2]{Dies ist eine Teilaufgabe.}
    \teilaufgabe[1]{\blindtext}
    \grid
\end{Aufgabe}
\begin{Aufgabe}[1]
    Eine Aufgabe mit grid.
    \grid
\end{Aufgabe}

\begin{Aufgabe}[1]
    Eine Aufgabe mit 4 cm langem grid.
    \grid[4]


\end{Aufgabe}


\begin{minipage}[t]{0.5\textwidth}
   \begin{Aufgabe}[8]
      Geht das? \\

      Und das?
      \teilaufgabe{Dies ist eine Teilaufgabe. Sie geht ein bisschen länger, sodass man das Alignment sieht denke ich. }
      \teilaufgabe{Dies ist eine Teilaufgabe.\\
      
      Sie geht ein bisschen länger, sodass man das Alignment sieht denke ich. }
   \end{Aufgabe}
\end{minipage}
\begin{minipage}[t]{0.5\textwidth}
    Ein dünnes Grid für 3 cm: \grid[3] 
   \end{minipage}



% \begin{Aufgabe}
%     \blindtext
% \end{Aufgabe}
% \begin{Aufgabe}
% Aufgabe mit einer Lösung. 
% \solution{Das sollte jetzt die umfassende Lösung zu der Aufgabe sein.}
% \end{Aufgabe}
% \printsols[3]

% \begin{Aufgabe}
%     \blindtext
%     \solution{Die noch umfassendere, aber leider nicht vollst}
% \end{Aufgabe}
% \printsols[last]
% \begin{Aufgabe}
%     \blindtext
% \end{Aufgabe}

% \printsols


\end{document}