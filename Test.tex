\documentclass[a4paper]{scrartcl}
\usepackage{schooltex}
\declarePath{.}
\usepackage{blindtext}
\usepackage[german]{babel}
\setkomafont{captionlabel}{\itshape}
\usepackage{lmodern}
\usepackage[headsepline, footsepline]{scrlayer-scrpage}
\pagestyle{scrheadings}
\usepackage[bottom=2.5cm, top=22mm, right = 2.5cm]{geometry}
\usepackage{hyperref}

\newcommand{\fach}{Physik}

\ifoot{Baier}
\ofoot{\fach}
\cfoot{}

%\uselengthunit{cm}
\newcommand{\Datum}{xx.xx.23}
\begin{document}


% \begin{SVerlauf}
%     10' & Ein Beispiel & Dies ist ein Beispiel & \lsg & Tafel \\ 
%     10' & Ein zweites Beispiel & Dies ist ein zweites  Beispiel Dies ist ein zweites  Beispiel Dies ist ein zweites  Beispiel Dies ist ein zweites  Beispiel  & \lsg & Tafel \\ 
%     10' & Ein zweites Beispiel & Dies ist ein zweites  Beispiel & \lsg & Tafel  \\ 
%     10' & Ein zweites Beispiel & Dies ist ein zweites  Beispiel & \lsg & Tafel \\ 
% \end{SVerlauf}
\begin{plan}
10' & Ein Beispiel & Dies ist ein Beispiel & \lsg & Tafel \\ 
10' & Ich teste das: &Wenn ich hier jetzt einen rießigen Text hineinschreibe, der im prinzip die gesamte
Information über die Stunde enthält, wird das dann etwas?  & \lsg & Tafel, Dokumentenkamera \\ 
10' & Ein zweites Beispiel & Dies ist ein zweites  Beispiel & \lsg & Tafel  \\ 
10' & Ein zweites Beispiel & Dies ist ein zweites  Beispiel & \lsg & Tafel \\ 
\end{plan}





\begin{Aufgabe}
    \teilaufgabe{Das ist ein Text  \\  mit umbruch22uizt
    }
\end{Aufgabe}

\end{document}