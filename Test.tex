\documentclass[a4paper]{scrartcl}
\usepackage{schooltex}
\usepackage{blindtext}
\usepackage[german]{babel}
\usepackage{hyperref}
\declarePath{.}
\uselengthunit{cm}
\begin{document}
\begin{SVerlauf}{Thema}
10' & Einstieg & Das wäre der Einstieg & & \ze
\end{SVerlauf}


\begin{SVerlauf}{Thema}
    10' & Einstieg & Das wäre der Einstieg & & \ze
\end{SVerlauf}

    \tafel{\kap{Ein Kapitel}
        \uber{Ueberschrift}
    Das ist ein Tafelbild.
    }

\newpage
\begin{Aufgabe}
Test Penis
\end{Aufgabe}

\begin{Aufgabe}
Das ist ein Test um zu schauen, ob es funktioniert!
\grid
\end{Aufgabe}
\begin{Aufgabe}
    Das ist ein weiterer Test für die Funktion dieses Dings!
    \\
    hoakh
    lhafkh\\
    \\
    \\

    \grid
\end{Aufgabe}


\begin{Aufgabe}
    \blindtext
    \grid[4]
\end{Aufgabe}

%\image[source = www.example.de]{Atwood2}
%\image[width = 6cm, caption = Eine Abbildung mit Caption und Quelle, source = www.example.de]{Atwood2}
%\image[width = 6cm]{Atwood2}
%\image[caption = Ohne Quelle{,} aber mit Caption]{Atwood2}
%\multiimage[source = www.expamle.com]{Atwood2,Atwood2, Atwood2, Atwood2, Atwood2}
%\multiimage[caption = Diesmal auch mit Caption , source = www.expamle.com]{Atwood2,Atwood2, Atwood2, Atwood2, Atwood2}
%\multiimage[caption = Mit caption]{Atwood2,Atwood2, Atwood2, Atwood2, Atwood2}
%blindtext
%\image[width = 10cm, source = examplequelle, position = l, label = fig:Label]{example-image-a}
%\blindtext
%In Abbildung \ref{fig:Label} sehen wir, dass wir hier ein Label definiert haben. Ich hoffe es klappt.
%\AB[]{UebungEnergieundLeistung}
%\AB[small]{UebungEnergieundLeistung}
%\AB[smalllandscape]{UebungEnergieundLeistung}
%\newpage


\end{document}