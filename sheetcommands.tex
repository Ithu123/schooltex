%%%Counter
\newcounter{Aufgabe}
\newcounter{Loesung}
%\setcounter{Loesung}{1}

\newcounter{TA}
\setcounter{TA}{1}

%\newcounter{TL}
%\setcounter{TL}{1}

\newcount\myvar


%%% Angaben für Symbolzahlen
\ProvideTextCommandDefault{\schriftlich}{1}
\ProvideTextCommandDefault{\mitTR}{2}
\ProvideTextCommandDefault{\ohneTR}{3}
\ProvideTextCommandDefault{\Aufg}{Aufgabe\,}

%Stufenaufgaben
\ProvideTextCommandDefault{\einS}{991}
\ProvideTextCommandDefault{\zweiS}{992}
\ProvideTextCommandDefault{\dreiS}{993}


%%%Hier fängt die Aufgabenumgebung an


\makeatletter
\define@key{AufgabeKeys}{symbol}{%
    \ifcase#1
        \def\margin_symbol{}% 0
    \or
        \def\margin_symbol{\rlap{\protect\makebox[-2em]{\includegraphics[height=8pt]{Icons/contract}}}}% 1
    \or
        \def\margin_symbol{\rlap{\protect\makebox[-2em]{\includegraphics[height=8pt]{Icons/calc}}}}% 2
    \or
        \def\margin_symbol{\rlap{\protect\makebox[-2em]{\includegraphics[height=8pt]{Icons/nocalc}}}}% 3
    \or
        \def\margin_symbol{\rlap{\protect\makebox[-2em]{\includegraphics[height=8pt]{Icons/calc} \includegraphics[height=8pt]{Icons/contract}}}}% 12
    \or
        \def\margin_symbol{\rlap{\protect\makebox[-2em]{\includegraphics[height=8pt]{Icons/calc} \includegraphics[height=8pt]{Icons/contract}}}}% 21
    \or
        \def\margin_symbol{\rlap{\protect\makebox[-2em]{\includegraphics[height=8pt]{Icons/nocalc} \includegraphics[height=8pt]{Icons/contract}}}}% 13
    \or
        \def\margin_symbol{\rlap{\protect\makebox[-2em]{\includegraphics[height=8pt]{Icons/nocalc} \includegraphics[height=8pt]{Icons/contract}}}}% 31
    \or
        \def\margin_symbol{\rlap{\protect\makebox[-3em]{$\bigstar$}}}% 991
    \or
        \def\margin_symbol{\rlap{\protect\makebox[-3em]{$\bigstar\bigstar$}}}% 992
    \or
        \def\margin_symbol{\rlap{\protect\makebox[-3em]{$\bigstar\bigstar\bigstar$}}}% 993
    \else
        \def\margin_symbol{}% Default
    \fi
}


\newlength\GesamtP	% Die Gesamtpunktzahl
\newlength\AufgP 	% Die Punkte innerhalb einer Aufgabe
\xdef\Punkte{0}		% Dass es bei der ersten Aufgabe nicht heißt, das gibt es nicht
\newlength\ZwischenP

\newlength{\pointspace}     % Space before the points
\setlength{\pointspace}{0.8cm}
\marginparsep8mm

\newcounter{AufgabeCounter}
\newcounter{SolutionCounter}[AufgabeCounter] % Use the same counter for solutions, reset with exercises

\NewDocumentEnvironment{Aufgabe}{O{0} O{0}}{%
    \setkeys{AufgabeKeys}{symbol=#2}% Set the symbol based on the argument
    \setlength{\GesamtP}{\Punkte pt} %Gesamtpunkte aufrufen
    \setlength{\pointspace}{0cm} % restor from teilaufgabe

    % Here begins the actual exercise
    \refstepcounter{AufgabeCounter}
    \paragraph{\Aufg\, \theAufgabe} \setlength{\AufgP}{#1 pt}  \setcounter{TA}{1} \refstepcounter{Aufgabe}  
}{%
\ifdim\AufgP=0pt
    \relax
\else
    \vspace{\pointspace}  \marginpar{\vspace{-4mm}\rule{0.8cm}{0.5pt}/\strip@pt\AufgP} 
% Punktzahlrechner am Ende
    \addtolength{\GesamtP}{\AufgP}						% Die neue Gesamtpunktzahl angeben
    \setlength{\AufgP}{0pt}								% Aufgabenpunkte für die nächste Aufgabe zurücksetzen
    \xdef\Punkte{\strip@pt\GesamtP} 					% Gesamtpunkte Global Speichern, xdef expandiert das Makro
\fi

    }





\ProvideDocumentCommand{\teilaufgabe}{ O{0} m}{%
    \setlength{\pointspace}{0.8cm}
    \setlength{\parskip}{0.3em}
    \begin{addmargin}[1em]{0em}
        \parbox[t]{0.6cm}{\textbf{\alph{TA})}}\refstepcounter{TA}\parbox[t]{0.9\linewidth}{#2}
    \end{addmargin}
    \ifnum#1 = 0
    \relax
    \else
    \marginpar{\vspace{-4mm}(#1 P.)} % Punktanzahl für die Teilaufgabe am Rand
    \addtolength{\AufgP}{#1 pt}	% Am Ende wird auf die Punktzahl die Punktzahl der Teilaufgabe dazugezählt
    \fi
    }
% Solution environment
\newenvironment{Loesung}{%
    \paragraph{Lösung Aufgabe \ref{aufgabe:\theSolutionCounter}}
    \refstepcounter{SolutionCounter}
    \label{loesung:\theSolutionCounter}
}{}
\makeatother










\newcommand{\Aufgabeauf}[1]{
    \myvar=#1

    \ifnum\myvar=1 % Ist die Variable 1?
        \def\margin_symbol{\rlap{\protect\makebox[-2em]{\includegraphics[height=8pt]{Icons/contract}}}}
    \else \relax % Wenn nicht, bleib locker
    \fi


    %\mitTR
    \ifnum\myvar=2
        \def\margin_symbol{\rlap{\protect\makebox[-2em]{\includegraphics[height=8pt]{Icons/calc}}}}
    \else \relax
    \fi

    %\ohneTR
    \ifnum\myvar=3
        \def\margin_symbol{\rlap{\protect\makebox[-2em]{\includegraphics[height=8pt]{Icons/nocalc}}}}
    \else \relax
    \fi

    %Combo!schriftlich mitTR
    \ifnum\myvar=12
        \def\margin_symbol{\rlap{\protect\makebox[-2em]{\includegraphics[height=8pt]{Icons/calc} \includegraphics[height=8pt]{Icons/contract}}}}
    \else \relax
    \fi
    \ifnum\myvar=21
        \def\margin_symbol{\rlap{\protect\makebox[-2em]{\includegraphics[height=8pt]{Icons/calc} \includegraphics[height=8pt]{Icons/contract}}}}
    \else \relax
    \fi
    %Combo! Schriftlich ohne TR
    \ifnum\myvar=13
        \def\margin_symbol{\rlap{\protect\makebox[-2em]{\includegraphics[height=8pt]{Icons/nocalc} \includegraphics[height=8pt]{Icons/contract}}}}
    \else \relax
    \fi
    \ifnum\myvar=31
        \def\margin_symbol{\rlap{\protect\makebox[-2em]{\includegraphics[height=8pt]{Icons/nocalc} \includegraphics[height=8pt]{Icons/contract}}}}
    \else \relax
    \fi

    \ifnum\myvar=991
        \def\margin_symbol{\rlap{\protect\makebox[-3em]{$\bigstar$}}}
    \else \relax
    \fi
    \ifnum\myvar=992
        \def\margin_symbol{\rlap{\protect\makebox[-3em]{$\bigstar\bigstar$}}}
    \else \relax
    \fi
    \ifnum\myvar=993
        \def\margin_symbol{\rlap{\protect\makebox[-3em]{$\bigstar\bigstar\bigstar$}}}
    \else \relax
    \fi
    \paragraph{\margin_symbol\Aufg \theAufgabe}  \setcounter{TA}{1} \refstepcounter{Aufgabe}  
}

\newcommand{\Aufgabezu}{ }

