%%%Counter
\newcounter{Aufgabe}
\newcounter{Loesung}



\newcounter{TA}
\setcounter{TA}{1}

\newcount\myvar


%%% Angaben für Symbolzahlen
\ProvideTextCommandDefault{\schriftlich}{1}
\ProvideTextCommandDefault{\mitTR}{2}
\ProvideTextCommandDefault{\ohneTR}{3}

%Stufenaufgaben
\ProvideTextCommandDefault{\einS}{991}
\ProvideTextCommandDefault{\zweiS}{992}
\ProvideTextCommandDefault{\dreiS}{993}


%%%Hier fängt die Aufgabenumgebung an
\makeatletter
\define@key{AufgabeKeys}{symbol}{%
    \ifcase#1
        \def\margin_symbol{}% 0
    \or
        \def\margin_symbol{\rlap{\protect\makebox[-2em]{\includegraphics[height=8pt]{Icons/contract}}}}% 1
    \or
        \def\margin_symbol{\rlap{\protect\makebox[-2em]{\includegraphics[height=8pt]{Icons/calc}}}}% 2
    \or
        \def\margin_symbol{\rlap{\protect\makebox[-2em]{\includegraphics[height=8pt]{Icons/nocalc}}}}% 3
    \or
        \def\margin_symbol{\rlap{\protect\makebox[-2em]{\includegraphics[height=8pt]{Icons/calc} \includegraphics[height=8pt]{Icons/contract}}}}% 12
    \or
        \def\margin_symbol{\rlap{\protect\makebox[-2em]{\includegraphics[height=8pt]{Icons/calc} \includegraphics[height=8pt]{Icons/contract}}}}% 21
    \or
        \def\margin_symbol{\rlap{\protect\makebox[-2em]{\includegraphics[height=8pt]{Icons/nocalc} \includegraphics[height=8pt]{Icons/contract}}}}% 13
    \or
        \def\margin_symbol{\rlap{\protect\makebox[-2em]{\includegraphics[height=8pt]{Icons/nocalc} \includegraphics[height=8pt]{Icons/contract}}}}% 31
    \or
        \def\margin_symbol{\rlap{\protect\makebox[-3em]{$\bigstar$}}}% 991
    \or
        \def\margin_symbol{\rlap{\protect\makebox[-3em]{$\bigstar\bigstar$}}}% 992
    \or
        \def\margin_symbol{\rlap{\protect\makebox[-3em]{$\bigstar\bigstar\bigstar$}}}% 993
    \else
        \def\margin_symbol{}% Default
    \fi
}


\newlength\GesamtP	% Die Gesamtpunktzahl
\newlength\AufgP 	% Die Punkte innerhalb einer Aufgabe
\xdef\Punkte{0}		% Dass es bei der ersten Aufgabe nicht heißt, das gibt es nicht
\newlength\ZwischenP

\newlength{\pointspace}     % Space before the points
\setlength{\pointspace}{0cm}
% \marginparsep8mm

\newcounter{AufgabeCounter}
\newcounter{SolutionCounter}[AufgabeCounter] % Use the same counter for solutions, reset with exercise



% Aufgabenumgebung
\ProvideDocumentEnvironment{Aufgabe}{O{0} O{0}}{%
    \setkeys{AufgabeKeys}{symbol=#2}% Set the symbol based on the argument
    \setlength{\GesamtP}{\Punkte pt} %Gesamtpunkte aufrufen
    \setlength{\pointspace}{0cm} % restor from teilaufgabe

    % Here begins the actual exercise
    \refstepcounter{AufgabeCounter}
    \paragraph{Aufgabe\, \theAufgabe} \setlength{\AufgP}{#1 pt}  \setcounter{TA}{1} \refstepcounter{Aufgabe}
\ifdim\AufgP=0pt
        \relax
\else
        \nopagebreak
        \leavevmode\marginnote{\rule{0.8cm}{0.5pt}/\strip@pt\AufgP}[\pointspace]
        %\vspace{\baselineskip}
        % Punktzahlrechner am Ende
        \addtolength{\GesamtP}{\AufgP}						% Die neue Gesamtpunktzahl angeben
        \setlength{\AufgP}{0pt}								% Aufgabenpunkte für die nächste Aufgabe zurücksetzen
        \xdef\Punkte{\strip@pt\GesamtP} 					% Gesamtpunkte Global Speichern, xdef expandiert das Makro
\fi
    }



% Teilaufgabenumgebung

\ProvideDocumentCommand{\teilaufgabe}{O{0} m}{%
    \setlength{\pointspace}{0.8cm}
    \setlength{\parskip}{0.3em}
    \begin{addmargin}[1em]{0em}
        \parbox[t]{0.6cm}{\textbf{\alph{TA})}}
        \refstepcounter{TA}
        \parbox[t]{0.9\linewidth}{#2}
        \ifnum#1 = 0
            Wir sind am Ende
            \relax
        \else
            \marginpar{\vspace{-4mm}(#1 P.)} % Punktanzahl für die Teilaufgabe am Rand
            \addtolength{\AufgP}{#1 pt}	% Am Ende wird auf die Punktzahl die Punktzahl der Teilaufgabe dazugezählt
        \fi
    \end{addmargin}
        }

   
   
% Material Command: This command should take a nbumber of materials for the excersize, manily tables and pictures 
\newlength{\materialwidth}
\makeatletter
\define@key{materials}{caption}{\def\materials@caption{#1}} % Define a key for the caption
\define@key{materials}{subcaptions}{\def\materials@subcaptions{#1}} % Define a key for the caption
\define@key{materials}{label}{\def\materials@label{#1}} % Define a key for the label
\define@key{materials}{rescale}{\def\rescale{#1}} % Define a key for the rescale each image
\define@key{materials}{source}{\def\materials@source{#1}}
\define@key{materials}{widtheach}{\def\materials@widtheach{#1}}



\ProvideDocumentCommand{\materials}{O{} m}{ % Optional arguments are provided as a key-value list
  \setkeys{materials}{caption=, label=, rescale = 1.0, source = ,widtheach = ,subcaptions = , #1} % Set default values and apply user-provided values
  
  %Look for captions
  \handleImageCaption{\materials@source}{\materials@caption}
  \handlelabel{\materials@label}
  \readlist\figruelist{#2} % Read and store the elements of the list


  % Look for subcaptions
  \ifdefempty{\materials@subcaptions}{ 
    \relax
  }{  
    \readlist\captionlist{\materials@subcaptions}
    }
  % Calculate the width of each subfigure based on the number of elements and rescale factor
\ifdefempty{\materials@widtheach}{
  \setlength{\materialwidth}{0.8\dimexpr(\rescale\linewidth/\figruelistlen)\relax}
  }{
  \setlength{\materialwidth}{\materials@widtheach}
  }
  \begin{figure}[h!]
    \centering
    \foreachitem\i\in\figruelist{
        \subcaptionbox{\captionlist[\icnt]}[\materialwidth]{\includegraphics[width = \materialwidth]{\i}}
        \stepcounter{captioncounter}
    }
    \setcounter{captioncounter}{1}
    \imagecaption 
    \mylabel % Use the provided label or the default label for the figure label
  \end{figure}
}
\makeatother




% Solution environment


% Solutions
\newcommand{\solution}[1]{\expandafter\gdef\csname storedsolution\romannumeral\value{Aufgabe}\endcsname{#1} \setcounter{TA}{1}}
\newcommand{\printsol}[1]{\paragraph{Lösung Aufgabe \, #1} \csname storedsolution\romannumeral #1\endcsname}

\makeatletter
\ProvideDocumentCommand{\printsols}{o}
{ 
    \IfNoValueTF{#1}
    {
        % If no argument is provided, print all solutions
        \foreach \i in {1,2,...,\value{Aufgabe}} {
            \@ifundefined{storedsolution\romannumeral\i}{
                % Solution is undefined, do nothing
            }{  
                \paragraph{Lösung zu Aufgabe \i:}
                \csname storedsolution\romannumeral\i\endcsname
            }
        }
    }
    {
        % Check if the argument is "last"
        \ifstrequal{#1}{last}{
            % Print the last solution
            \edef\lastsolution{\number\numexpr\value{Aufgabe}\relax}
            \@ifundefined{storedsolution\romannumeral\lastsolution}{
                % Solution is undefined, provide a message
                \paragraph{Lösung zur letzten Aufgabe:}
                Solution not found.
            }{  
                \paragraph{Lösung zu Aufgabe \lastsolution:}
                \csname storedsolution\romannumeral\lastsolution\endcsname
            }
        }{
            % Check if the argument is a number
            \ifnum\pdfstrcmp{#1}{\number\numexpr#1}=0
                % If it's a number, print the solution for that number
                \@ifundefined{storedsolution\romannumeral#1}{
                    % Solution is undefined, provide a message
                    \paragraph{Lösung zu Aufgabe #1:}
                    Solution not found.
                }{  
                    \paragraph{Lösung zu Aufgabe #1:}
                    \csname storedsolution\romannumeral#1\endcsname
                }
            \else
                % If it's not "last" or a number, print an error message
                \paragraph{Error:} Invalid argument.
            \fi
        }
    }
}
\makeatother



%%% Ende der Klassenarbeit wird definiert

\makeatletter
\newcommand{\Ende}{\vspace{3em}\begin{center} \textbf{Viel Erfolg!} \end{center} \textbf{Erreichte Gesamtpunktzahl:}\rule{1.5cm}{0.5pt}/\Punkte   \newline \newline \newline
\textbf{Note:}\rule{1.5cm}{0.5pt} \hfill \textbf{In Worten:} \rule{5cm}{0.5pt}\\
\vspace{1cm}
\makeatother

\noindent
\begin{tblr}{l X l}
 \hspace{3cm}   & & \hspace{3cm} \\\cline{1-1}\cline{3-3}
 Ort, Datum     & & Unterschrift
\end{tblr} } % Das Ende der KA


%%% Header
\newcommand{\header}[2]{
%\vspace{-1em}
\textbf{Bearbeitungszeit:} #1$\,\mathrm{min}$\hfill \Datum \\

\textbf{Zugelassene Hilfsmittel:} #2\\
\\
\textbf{Schreibe unter jede Textaufgabe einen Antwortsatz.}
\hrule}






\newcommand{\Aufgabeauf}[1]{
    \myvar=#1

    \ifnum\myvar=1 % Ist die Variable 1?
        \def\margin_symbol{\rlap{\protect\makebox[-2em]{\includegraphics[height=8pt]{Icons/contract}}}}
    \else \relax % Wenn nicht, bleib locker
    \fi


    %\mitTR
    \ifnum\myvar=2
        \def\margin_symbol{\rlap{\protect\makebox[-2em]{\includegraphics[height=8pt]{Icons/calc}}}}
    \else \relax
    \fi

    %\ohneTR
    \ifnum\myvar=3
        \def\margin_symbol{\rlap{\protect\makebox[-2em]{\includegraphics[height=8pt]{Icons/nocalc}}}}
    \else \relax
    \fi

    %Combo!schriftlich mitTR
    \ifnum\myvar=12
        \def\margin_symbol{\rlap{\protect\makebox[-2em]{\includegraphics[height=8pt]{Icons/calc} \includegraphics[height=8pt]{Icons/contract}}}}
    \else \relax
    \fi
    \ifnum\myvar=21
        \def\margin_symbol{\rlap{\protect\makebox[-2em]{\includegraphics[height=8pt]{Icons/calc} \includegraphics[height=8pt]{Icons/contract}}}}
    \else \relax
    \fi
    %Combo! Schriftlich ohne TR
    \ifnum\myvar=13
        \def\margin_symbol{\rlap{\protect\makebox[-2em]{\includegraphics[height=8pt]{Icons/nocalc} \includegraphics[height=8pt]{Icons/contract}}}}
    \else \relax
    \fi
    \ifnum\myvar=31
        \def\margin_symbol{\rlap{\protect\makebox[-2em]{\includegraphics[height=8pt]{Icons/nocalc} \includegraphics[height=8pt]{Icons/contract}}}}
    \else \relax
    \fi

    \ifnum\myvar=991
        \def\margin_symbol{\rlap{\protect\makebox[-3em]{$\bigstar$}}}
    \else \relax
    \fi
    \ifnum\myvar=992
        \def\margin_symbol{\rlap{\protect\makebox[-3em]{$\bigstar\bigstar$}}}
    \else \relax
    \fi
    \ifnum\myvar=993
        \def\margin_symbol{\rlap{\protect\makebox[-3em]{$\bigstar\bigstar\bigstar$}}}
    \else \relax
    \fi
    \paragraph{\margin_symbol\Aufg \theAufgabe}  \setcounter{TA}{1} \refstepcounter{Aufgabe}  
}

\newcommand{\Aufgabezu}{ }

