%Skript Commands 
%Arbeitsblatt einbilden

\newcommand{\declarePath}[1]{\providecommand{\Path}{#1}}
% Define a command that checks if the Path is set
\define@key{AB}{format}{\def\image@format{#1}}


\makeatletter
\NewDocumentCommand{\AB}{ O{} m }{%
  \@ifundefined{Path}{%
    \PackageError{schooltex}{Path not set. Please set the \string\Path macro.}{}%
  }{%
    \begingroup
    \IfStrEq{#1}{small}{%
      \includepdf[pages=-, nup=1x2, doublepages, landscape, frame]{\Path/#2/#2}%
    }{%
      \IfStrEq{#1}{smalllandscape}{%
        \includepdf[pages=-, nup=1x2, doublepages, frame]{\Path/#2/#2}%
      }{%
          \IfStrEq{#1}{landscape}{%
          \includepdf[pages=-, landscape]{\Path/#2/#2}%
        }{%
          \includepdf[pages=-]{\Path/#2/#2}% Default case
        }%
      }%
    }%
    \addcontentsline{toc}{subsection}{#2}%
    \endgroup
  }%
}
\makeatother
    
%Stundenverlauf
%Methoden
\ProvideTextCommandDefault{\lsg}{L-S G.}
\ProvideTextCommandDefault{\lv}{LV}
\ProvideTextCommandDefault{\ea}{EA}
\ProvideTextCommandDefault{\pa}{PA}
\ProvideTextCommandDefault{\ga}{GA}



\NewDocumentEnvironment{SVerlauf}{ O{12} g }
{% Begin code
    \subsubsection{Stundenverlauf%
        \IfValueT{#2}{\ #2}}
    \begin{adjustbox}{width=\textwidth,center}
        \tabular{|c|l|p{#1 cm}|l|l|}
            \hline
            \textbf{Zeit} & \textbf{Thema} &\textbf{Ausführung/Plan} &\textbf{Methode}  & \textbf{Medien}  \\
            \hline
}
{% End code
        \endtabular
    \end{adjustbox}
    \addvspace{\baselineskip}
}
\NewTableCommand\te{ \\ \hline[0.1em,red5]}

\NewDocumentEnvironment{plan}{O{Stundenverlauf} +b}{
  \par 
  \vspace{\baselineskip}
  \noindent
  \scriptsize
  \section{#1}
\begin{tblr}{hlines,vlines,row{1}={gray9}, colspec = {Q[c] X[2,l] X[5,l] Q[l] Q[l]}, width = \textwidth, measure = vbox, stretch = 1.2}
  \textbf{Zeit} & \textbf{Thema} & \textbf{Ausführung/Plan} & \textbf{Methode} &{ \textbf{Medien}, \\\textbf{Material} }\\
#2
\end{tblr}

}{\addvspace{\baselineskip}}




\newcommand{\umbruch}[1]{\begin{tabular}[c]{@{}l@{}} #1 \end{tabular}}

% Tafel
%Überscrift
\newenvironment{Ueber}[1]{
  \begin{scshape}
    \underline{#1}
  }{
    \end{scshape}\newline
  }

\newcounter{kap} %Counter für das Kapitel
\newcounter{unterkap}[kap] %Counter für die Stunde (das Unterthema)
\newcounter{unterunterkap}[unterkap]


\providecommand\tafel[1]{\subsubsection{Tafelbild}
\vspace{1em}\fbox{\parbox{\textwidth}{#1}}\vspace{1em}}

% Tafelbildüberschriften
\providecommand\kap[1]{\stepcounter{kap} \begin{scshape}\underline{\thekap \hspace{0.7em} #1} \end{scshape}  \vspace{1em}\\}
\providecommand\uber[1]{\refstepcounter{unterkap}\underline{\thekap .\theunterkap \hspace{0.7em} #1}  \vspace{1em}  \\}
\providecommand\uuber[1]{\refstepcounter{unterunterkap}\underline{\thekap .\theunterkap .\theunterunterkap \hspace{0.7em} #1} \vspace{1em}\\}

% Ohne Nummer
\providecommand{\kapo}[1]{ \begin{scshape}\underline{#1} \end{scshape}  \vspace{1em}\\}
\providecommand{\ubero}[1]{\underline{#1}  \vspace{1em}  \\}
\providecommand{\uubero}[1]{\underline{#1} \vspace{1em}\\}

